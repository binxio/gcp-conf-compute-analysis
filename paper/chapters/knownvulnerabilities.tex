%!TEX root = ../main.tex

\section*{Confidential Compute - Known Vulnerabilities}
\addcontentsline{toc}{section}{Confidential Compute - Known Vulnerabilities}


This table is in addition to the table provided bij AMD previously mentioned.
This table only includes the literature mentioned in this chapter. 
The main conclusion is that the known vulnerabilities 
are to be mitigated in the SEV-SNP. 
The development team of AMD is really transparent on the found vulnerabilities 
and in many papers thanked for their collaboration 
on confirming the found vulnerabilities. 

\begin{tabular}{ |l|c|c|c| }
   & \textbf{SEV} & \textbf{SEV-ES} & \textbf{SEV-SNP}\\
  \hline

  \textbf{to be sorted} 
  & \cite{hetzelt_security_2017,du_secure_2017,li_tlb_2021}
  & \cite{li_tlb_2021} 
  & \\

  \textbf{Physical attacks} 
  & \cite{li_exploiting_2019,buhren_insecure_2019,buhren_one_2021} 
  & \cite{wilke_sevurity_2020,buhren_one_2021} 
  & \cite{buhren_one_2021}\\

  \textbf{Software attacks} 
  & \cite{morbitzer_severed_2018,hetzelt_via_2021}
  & \cite{hetzelt_via_2021} 
  & \\

  \textbf{Side-channel attacks} 
  & \cite{li_crossline_2021,li_cipherleaks_2021,mestas_exploitation_2021} 
  & \cite{li_crossline_2021,li_cipherleaks_2021,mestas_exploitation_2021} 
  & \\

  \textbf{Denial-of-service attacks} & & & \\

\end{tabular}

There are two authors that are very active in the discovery of the vulnerabilities. 
These two authors are: 
\begin{enumerate}
  \item Mengyuan Li \\
  from the Ohio State University.  
  \item Robert Buhren \\
  from T-Lab TU  Berlin.
\end{enumerate}
