%!TEX root = ../main.tex

\subsection{\cite{li_cipherleaks_2021} - CipherLeaks: Breaking Constant-time Cryptography on AMD SEV via the Ciphertext Side Channel}

\textbf{CipherLeaks: Breaking Constant-time Cryptography on AMD SEV via the Ciphertext Side Channel}

\subsubsection*{Abstract \cite{li_cipherleaks_2021}}
“AMD’s Secure Encrypted Virtualization (SEV) is a hardware extension available in AMD’s EPYC server processors to support confidential cloud computing. While various prior studies have demonstrated attacks against SEV by exploiting its lack of encryption in the VM control block or the lack of integrity protection of the encrypted memory and nested page tables, these issues have been addressed in the subsequent releases of SEV-Encrypted State (SEV-ES) and SEV-Secure Nested Paging (SEV-SNP).

In this paper, we study a previously unexplored vulnerability of SEV, including both SEV-ES and SEV-SNP. The vulnerability is dubbed ciphertext side channels, which allows the privileged adversary to infer the guest VM’s execution states or recover certain plaintext. To demonstrate the severity of the vulnerability, we present the CIPHERLEAKS attack, which exploits the ciphertext side channel to steal private keys from the constant-time implementation of the RSA and the ECDSA in the latest OpenSSL library“

\subsubsection*{Conclusion \cite{li_cipherleaks_2021}}
“This paper describes the ciphertext side channel on SEV (including SEV-ES and SEV-SNP) processors. The root causes of the side channel are two-fold: First, SEV uses XEX mode of encryption with a tweak function of the physical addresses, so that the one-to-one mapping between the ciphertext and plaintext of the same address is preserved. Second, the VM memory is readable by the hypervisor, allowing it to monitor the changes of the ciphertext blocks. The paper demonstrates the CIPHERLEAKS attack that exploits the ciphertext sidechannel vulnerability to completely break the constant-time cryptography of OpenSSL when executed in SEV-ES VMs.”
