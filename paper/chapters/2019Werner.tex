%!TEX root = ../main.tex

\subsection{\cite{2019-werner} - The SEVerESt Of Them All: Inference Attacks Against Secure Virtual Enclaves}

\textbf{The SEVerESt Of Them All: Inference Attacks Against Secure Virtual Enclaves}


\subsubsection*{Abstract \cite{2019-werner}}
"The success of cloud computing has shown that the cost and convenience benefits of outsourcing infrastructure, platform, and software resources outweigh concerns about confidentiality. Still, many
businesses resist moving private data to cloud providers due to intellectual property and privacy reasons. A recent wave of hardware
virtualization technologies aims to alleviate these concerns by offering encrypted virtualization features that support data confidentiality of guest virtual machines (e.g., by transparently encrypting
memory) even when running on top untrusted hypervisors.

We introduce two new attacks that can breach the confidentiality
of protected enclaves. First, we show how a cloud adversary can
judiciously inspect the general purpose registers to unmask the
computation that passes through them. Specifically, we demonstrate
a set of attacks that can precisely infer the executed instructions
and eventually capture sensitive data given only indirect access to
the CPU state as observed via the general purpose registers. Second,
we show that even under a more restrictive environment — where
access to the general purpose registers is no longer available —
we can apply a different inference attack to recover the structure
of an unknown, running, application as a stepping stone towards
application fingerprinting. We demonstrate the practicality of these
inference attacks by showing how an adversary can identify different applications and even distinguish between versions of the same
application and the compiler used, recover data transferred over
TLS connections within the encrypted guest, retrieve the contents
of sensitive data as it is being read from disk by the guest, and inject
arbitrary data within the guest. Taken as a whole, these attacks
serve as a cautionary tale of what can go wrong when the state of
registers (e.g., in AMD’s SEV) and application performance data
(e.g., in AMD’s SEV-ES) are left unprotected. The latter is the first
known attack that was designed to specifically target SEV-ES.


\subsubsection*{Conclusion \cite{2019-werner}}
"To address cloud confidentiality, virtualization technologies have recently offered encrypted virtualization features that support transparent encryption of memory as a means of protection against
malicious tenants or even untrusted hypervisors. In this paper, we
examine the extent to which these technologies meet their goals.
In particular, we introduce a new class of inference attacks and
show how these attacks can breach the privacy of tenants relying
on secure encrypted virtualization technologies. As a concrete case
in point, we show how the security of the Secure Encrypted Virtualization (SEV) platform can be undermined. Additionally, we
show that even when additional state is encrypted (e.g., as proposed
under the SEV-ES extension where the state of general purpose registers is also encrypted), an adversary may still mount application
fingerprinting attacks, rendering those protections less effective
than first thought. We provide suggestions for mitigating the threat
posed by some of these attacks in the short term."