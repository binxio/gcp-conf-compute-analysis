%!TEX root = ../main.tex

\subsection{\cite{morbitzer_severed_2018} - SEVered: Subverting AMD’s Virtual Machine Encryption}

\textbf{SEVered: Subverting AMD’s Virtual Machine Encryption}

\subsubsection*{Abstract \cite{morbitzer_severed_2018}}
“AMD SEV is a hardware feature designed for the secure encryption of virtual machines. SEV aims to protect virtual machine memory not only from other malicious guests and physical attackers, but also from a possibly malicious hypervisor. This relieves cloud and virtual server customers from fully trusting their server providers and the hypervisors they are using. We present the design and implementation of SEVered, an attack from a malicious hypervisor capable of extracting the full contents of main memory in plaintext from SEV-encrypted virtual machines. SEVered neither requires physical access nor colluding virtual machines, but only relies on a remote communication service, such as a web server, running in the targeted virtual machine. We verify the effectiveness of SEVered on a recent AMD SEV-enabled server platform running different services, such as web or SSH servers, in encrypted virtual machines. With these examples, we demonstrate that SEVered reliably and efficiently extracts all memory contents even in scenarios where the targeted virtual machine is under high load.” 

\subsubsection*{Conclusion \cite{morbitzer_severed_2018}}
“We presented the design and implementation of SEVered, an attack that reliably extracts the full plaintext memory of VMs encrypted with AMD SEV from a malicious HV. The only major requirement for our method is the presence of a service in the VM, which provides a resource to the outside. Such services are usually easy to find, since VMs are typically and widely used in server contexts where they host web servers and other remotely accessible services. We demonstrated the feasibility of our approach by realizing a prototype on a recent AMD SEV-enabled platform. We evaluated the prototype with different services, namely the Apache and nginx web servers and an OpenSSH server. For every service, we considered various levels of concurrent accesses to evaluate SEVered under different, realistic load conditions. In all cases, we were able to efficiently identify the relevant resource of the target VM in memory by analyzing the VM’s memory access patterns from the HV. With the gained knowledge we were able to use the malicious HV to remap the resource to other memory pages and to iteratively request all the VM’s memory in reasonable time. As SEVered is independent of the specific service, our method can easily be adapted to a variety of different attack scenarios in practice. SEVered demonstrates that a malicious HV still remains able to extract sensitive data from its SEV-enabled guest VMs.”

