%!TEX root = ../main.tex

\section{GCP Titan (TPM)}
%\addcontentsline{toc}{section}{GCP Titan (TPM)}

All of Google Cloud runs on Google purpose built servers 
which contain a by google custom build chip\footnote{\url{https://en.wikipedia.org/wiki/Secure_cryptoprocessor}}, called Titan\footnote{\url{https://en.wikipedia.org/wiki/Titan_Security_Key}}. 

"Google's purpose-built chip to establish hardware root of trust 
for both machines peripherals on cloud infrastructure" - 
\cite{holzle_bolstering_2017}

According to Google,
 “Titan works to ensure that a machine boots from a known good state 
 using verifiable code, and establishes the hardware root of trust 
 for cryptographic operations in our data centers.” \cite{savagaonkar_titan_2017}.
 Titan is a 
 Trusted Platform Module\footnote{\url{https://en.wikipedia.org/wiki/Trusted_Platform_Module}} 
 (TPM).

At a high level, the Titan chip’s primary duties are to:
(1) Ensure authenticated software components (Secure Boot) and
(2) Establish a hardware root of trust (Machine Identity).

\begin{figure}[!ht]
    \centering
    \includegraphics[width=0.8\linewidth]{what-is-titan}
    \caption{What is Titan image from \cite{johnson_titan_2018}}
    \label{fig:what-is-titan}
\end{figure}

Let's explore how Titan performs these duties:

\subsection{Secure Boot}
Using public key cryptography\footnote{\url{https://opensource.googleblog.com/2020/12/opentitan-at-one-year-open-source.html}}\textsuperscript{,}\footnote{\url{https://opentitan.org/}}, 
the Titan chip validates the boot firmware from a known baseline using digital signatures. 
This ensures nothing at the firmware level has been tampered with, and the machine is trusted.

\subsection{Machine Identity}
Cryptographic keys are fused into the Titan chip during its creation. 
Using these fused keys and cryptography we can ensure the Titan chip is valid, 
creating a hardware base root of trust with which we can establish a trusted identity from.

For a much more in-depth technical explanation of these concepts, 
see this Google blog "Titan in depth: Security in plaintext" \cite{savagaonkar_titan_2017},
this presentation by Johnson at the secure enclave workshop at google \citep{johnson_titan_2018}, 
and this ebook on Secure boot \citep{yao_understanding_2021}.

\begin{figure}[!ht]
    \centering
    \includegraphics[width=0.9\linewidth]{titan-system-integration}
    \caption{Titan system integration image from \cite{yao_understanding_2021}}
    \label{fig:titan-system-integration}
\end{figure}


\subsection{Titan extended to Shielded VM’s}
The hardware integrity provided by the Titan chip is extended 
to the Virtual Machine (VM), by the concept of the Shielded VM 
\citep{leibl_gcp_2022}.

